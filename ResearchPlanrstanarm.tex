% NIH Grant Proposal for the Specific Aims and Research Plan Sections
%----------------------------------------------------------------------------------------
%	PACKAGES AND OTHER DOCUMENT CONFIGURATIONS
%----------------------------------------------------------------------------------------

\documentclass[11pt,notitlepage]{article}

% A note on fonts: As of 2013, NIH allows Georgia, Arial, Helvetica, and Palatino Linotype. LaTeX doesn't have Georgia or Arial built in; you can try to come up with your own solution if you wish to use those fonts. Here, Palatino & Helvetica are available, leave the font you want to use uncommented while commenting out the other one.

\usepackage[super]{natbib}
\usepackage{palatino} % Palatino font
%\usepackage{helvet} % Helvetica font
\renewcommand*\familydefault{\sfdefault} % Use the sans serif version of the font
\usepackage[T1]{fontenc}
\linespread{1.05} % A little extra line spread is better for the Palatino font
\usepackage{hyperref} % to allow hyperlinks to websites on the internet
\usepackage[hypcap]{caption} % to point to the top of the image
\usepackage{lipsum} % Used for inserting dummy 'Lorem ipsum' text into the template
\usepackage{amsfonts, amsmath, amsthm, amssymb} % For math fonts, symbols and environments
\usepackage{graphicx} % Required for including images
\usepackage{booktabs} % Top and bottom rules for tables
\usepackage{wrapfig} % Allows in-line images
\usepackage[labelfont=footnotesize]{caption} % Make figure numbering in captions bold
\usepackage[top=0.6in,bottom=0.6in,left=0.6in,right=0.6in]{geometry} % Reduce the size of the margin
\pagestyle{empty} % Remove page numbers

\hyphenation{ionto-pho-re-tic iso-tro-pic fortran} % Specifies custom hyphenation points for words or words that shouldn't be hyphenated at all

  % to reduce white space between SECTIONS
\usepackage[compact]{titlesec}
%\titlespacing{\part}{0pt}{5pt}{4pt}
%\titlespacing{\subsection}{0pt}{*0}{*0}
\titlespacing{\subsubsection}{-2pt}{*0}{*0}
%\titlespacing{\subparagraph}{-5pt}{*0}{*0}
%\titlespacing*{\subparagraph} {\parindent}{1ex plus 1ex minus .2ex}{0.5em}

  
  % to reduce white space between PARAGRAPHS
%\setlength{\parskip}{-2pt}
% \setlength{\parsep}{-2pt}

  % additional parameters
%\setlength{\headsep}{0pt}
%\setlength{\topskip}{0pt}
%\setlength{\topmargin}{0pt}
%\setlength{\topsep}{0pt}
%\setlength{\partopsep}{-2pt}
%\setlength{\parindent}{1cm}

% to reduce white space around figures
% \setlength{\textfloatsep}{0pt plus 0pt minus 0pt}

\begin{document}

\part*{Research Strategy}


\section*{Significance}

\subsection*{Big Models for Big Data}

\subsection*{Need for more complex models for EHR}

\subsubsection*{subsubsection on why}

\subsection*{Difficulty to fit complex models}

\subsection*{Stan is flexible and fast}

\subsection*{Modeling should be accessible}

\section*{Innovation}

\subsection*{H MC Algorithm is faster}

\section*{Approach}
Multidimensional statistical and computational methods for analyzing, inspecting, displaying, representing, parsing, and searching high-dimensional data

\subsection*{Preliminary work}
Funded through several mechanisms including the National Science Foundation and the National Institute of Health, the team submitting this proposal already developed and implemented many related algorithms and software packages. The proposed work is a direct continuation of the below described preliminary work of developing and implementing novel ground breaking algorithms for hierarchical modeling of complex data.
\subsubsection*{A novel algorithm for Bayesian inference: Hamiltonian Monte Carlo}
Dr. Gelman, Betancourt and collaborators developed Hamiltonian Monte Carlo (HMC) methods, a novel approach to computationally implement complex hierarchical Bayesian inference through Monte Carlo simulation. 
\subsubsection*{Open source computational implementation of HMC for diverse interfaces: Stan.} Drs. Gelman, Betancourt and Goodrich developed Stan, an open source multipurpose probabilistic programming language to build complex Bayesian models in several open source and commerical software including so far Stata, Mathlab, Python, Julia and R/Rstudio.  
\subsubsection*{Implementation in the open source software environment R/Rstudio: rstan} Drs. Goodrich, Gelman, Betancourt and collaborators developed Rstan, a software package to use Hamiltonian Monte Carlo algorithms the open source statistical software enviroment R/Rstudio.
\subsubsection*{Prototype software development: shinyStan and rstanArm} Drs. Goodrich, Andreae, Gelman and Betancourt and collaborators developed two additional prototype software package for the open source statistical software environment R/Rstudio, called (1) rstanarm and (2) shinystan.

\paragraph*{(1) rstanarm makes building advanced hierarchical Bayesian models accessible to data scientist} without the need for an understanding of the complexities underlying Hamiltonian Monte Carlo. This software package uses the same notation for model description as other widely accepted software packages for mixed modeling in R/Rstudio like lme4. 
\paragraph*{(2) shinystan is an interactive tool to explore and diagnose the output of Monte Carlo simulations}for Bayesian inference.  Also a package for R/Rstudio, shinystan  provides multidimensional statistical, graphical and computational tools for any analyzing, inspecting, displaying, representing, parsing, and searching high-dimensional MCMC output, but is optimized for HMC.

\subsection*{Team experience in complex hierarchical modeling in medicine and software development}

\paragraph*{Our team has extensive experience in hierarchical and complex modeling in medicine} and electronic medical records. Dr. Andreae published several systematic reviews and meta-analyses \cite{Andreae2013, Andreae2015, Carter2015}. Dr. Goodrich Dr. Hall is nationally recognized for the development and application of change point models in epidemiology and surveillance. Dr. Gong is leading an NIH funded trial to predict and improve respiratory outcomes after intubation based on real time electronic medical records.  Drs. Andreae, Goodrich and collaborators used the software prototype rstanarm and shinystan to build a multilevel hierarchical model to investigate health care disparities and quality of anesthesia delivery in the large National Clinical Outcomes Registry maintained by the American Society of Anesthesiology. Dr. Gelman is internationally recognized as a leader in hierarchical modeling with past and present funding and publication in pharmacodynamic modeling, ...

\paragraph*{Our team has a solid background in the development of complex statistical software} and in the visualization of model parameters, co-variance matrices and statistical data.  Drs. Goodrich and Gelman developed several widely cited and used software packages including the probabilistic programming language Stan, the basis for the proposed work. Dr. Goodrich and Andreae work together on the preliminary software packages shinystan and rstanarm.  

\subsubsection*{Best practices and proven methods for software design, construction, and implementation}
\subsubsection*{Mechanisms for incorporating user reported corrections into the software.}

\subsection*{Timeline}




\begin{table}[]
\resizebox{\textwidth}{!}{%
\begin{tabular}{llllll}
\hline
            & Year 1                             & Year 2                  & Year 3                       & Year 4             & Year 5       \\ \hline
Stan GLM    & meta-analysis                      & 3-4 level models        & \multicolumn{2}{l}{semi- and parametric survival} & change point \\
Shinystan   & Binary posterior predictive check & dependence plot         & tree depth plot              &                    &              \\
New Methods & Dependence plot                    & Multilevel model priors &                              &                    &              \\ \hline
\end{tabular}
}
\caption{Timeline: The above timeline outlines our targets and milestones over the five year grant period.}
\label{Timeline}
\end{table}

\subsection*{test}

%----------------------------------------------------------------------------------------
%	BIBLIOGRAPHY
%----------------------------------------------------------------------------------------

\newpage

% \nobibliography{K01_bibliography_24Feb15} % to not print a bibiography at the end.
\bibliography{Bibliography/bibliographyBD2K} % the file name cannot contain spaces
\bibliographystyle{Bibliography/nihunsrt} % Use the custom nihunsrt bibliography style included with the template

%----------------------------------------------------------------------------------------

\end{document}